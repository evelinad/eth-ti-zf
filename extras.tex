\chapter{Weitere Informationen}
\section{Hamiltonischer Kreis}
Ein \textbf{Hamiltonischer Kreis} eines Graphen $G$ ist ein geschlossener Weg, der jeden Knoten von $G$ genau einmal enthält.

\section{Traveling Salesman Problem (TSP)}
Fragestellung: Gegeben ist eine Menge von Städten und der Reisedistanz zwischen diesen Städten. Gesucht ist die kürzeste Route mit welcher alle Städte genau einmal besucht werden und mit der man wieder beim Anfang landet.

Es handelt sich um ein NP-vollständiges Problem.

Zur Lösung dieses Problems wird der kürzeste Hamiltonsche Kreis gesucht, der am gegebenen Startpunkt beginnt.

\section{Minimum Vertex Cover Problem (MIN-VCP)}
Gesucht wird die minimale Knotenmenge (\ref{sec:knotenueberdeckung}), die eine Knotenüberdeckung eines Graphen ist.

\section{Sprachklassen}
\subsection{Klassenübersicht}
\todo[inline]{Übersicht und Bedeutung}

\subsection{Sprachenzugehörigkeit}
\todo[inline]{Übersicht welche Sprachen zu welcher Klasse gehören}

\section{Formeln}
\subsection{Summenformel für Anzahl Programme}
\[
\sum_{i = k}^n 2^i = 2^{n+1} - 1 - \sum_{i = 0}^k 2^i
\]