\documentclass[a4paper,titlepage]{report}

%
% utf8 fix
%
\usepackage[utf8]{inputenc}

%
% amsthm
%
\usepackage{amsthm}

\theoremstyle{plain}
\newtheorem{lemma}{Lemma}[section]
\newtheorem{theorem}{Theorem}[section]
\newtheorem{corollary}{Korollar}[section]
\newtheorem{tipp}{Tipp}[section]

\theoremstyle{definition}
\newtheorem{definition}{Definition}[section]
\newtheorem{satz}{Satz}[section]
\newtheorem{hilfssatz}{Hilfssatz}[section]
\newtheorem{remark}{Bemerkung}[section]

%
% disable paragraph indentation
%
\usepackage{parskip}

%
% PGF/TikZ for state machines
%
\usepackage{
  pgf,
  tikz
}
\usetikzlibrary{
  arrows,
  automata
}


%
% other packages
%
\usepackage{
  amssymb,
  amsfonts,
  amsmath,
  enumitem,
  hyperref,
  todonotes
}

%
% listings
%
\usepackage{listings}

\lstset{
  language=Pascal,
  escapeinside={||}
}

%
% number sets
%
\newcommand{\R}{\mathbb{R}}
\newcommand{\Z}{\mathbb{Z}}
\newcommand{\N}{\mathbb{N}}
\newcommand{\Q}{\mathbb{Q}}
\newcommand{\C}{\mathbb{C}}
\newcommand{\F}{\mathbb{F}}
\newcommand{\E}{\mathbb{E}}
\newcommand{\LL}{\mathcal{L}}
\newcommand{\powerset}{\mathcal P}

%
% section settings
%
\setcounter{secnumdepth}{3}

%
% document
%

\begin{document}

% title
\title{Theoretische Informatik Zusammenfassung HS13}
\author{Gregor Wegberg}
\date{\today}
\maketitle

% index
\tableofcontents
\clearpage

% todo list
\listoftodos
\clearpage

% chapters
\chapter{Alphabete, Wörter, Sprachen}
Zur Darstellung von Daten werden Symbole verwendet. Diese wiederum werden zur Bildung von Wörtern genutzt. Und aus Wörtern bildet sich eine Sprache.

\section{Alphabet}
\begin{definition}
Eine endliche nichtleere Menge $\Sigma$ heisst \textbf{Alphabet}. Die Elemente von $\Sigma$ werden als \textbf{Buchstaben, Zeichen oder Symbole} bezeichnet.
\end{definition}

Häufig benötigte Alphabete:
\begin{itemize}
    \item $\Sigma_\text{bool} = \{0,1\}$
    \item $\Sigma_\text{lat} = \{a, b, c, \ldots, z\}$
    \item $\Sigma_\text{Tastatur} = \Sigma_\text{lat} \cup \{A, B, C, \ldots, \textvisiblespace, <, >, \ldots\}$ (Alphabet aller Zeichen auf einer Tastatur)
    \item $\Sigma_m = \{0, 1, 2, \ldots, m - 1\}$ (Alphabet für die $m$-adische Darstellung von Zahlen, $m \geq 1$)
    \item $\Sigma_\text{logic} = \{0, 1, x, (, ), \land, \lor, \lnot\}$ (Alphabet um Boole'sche Formeln darzustellen)
\end{itemize}

\section{Wort}
\subsection{Grundlagen}
Ein \textbf{Wort} wird aus Elementen eines zugrundeliegenden Alphabets gebildet.\\

\begin{definition}
Sei $\Sigma$ ein Alphabet. Ein \textbf{Wort} über $\Sigma$ ist eine endliche Folge von Buchstaben aus $\Sigma$.\\
\end{definition}

\begin{remark}
$\Sigma^*$ ist die Menge aller Wörter über dem Alphabet $\Sigma$. $\Sigma^+ = \Sigma^* - \{\lambda\}$, also die Menge aller Wörter ohne das leere Wort.\\
\end{remark}


\begin{definition}
Das \textbf{leere Wort} wird durch $\lambda$ (oft auch $\epsilon$) dargestellt und entspricht der leeren Folge.\\
\end{definition}

\begin{definition}
Die \textbf{Länge} eines Wortes $w$, bezeichnet durch $|w|$, ist die Länge der Folge, d.h. die Anzahl vorkommender Buchstaben in $w$.\\
\end{definition}

\begin{remark}
Das leere Wort $\lambda$ ist ein Wort über jedem Alphabet.\\
\end{remark}

\begin{definition}
Sei $w \in \Sigma^*$ und $a \in \Sigma$. Dann ist $|w|_a$ definiert als die Anzahl der Vorkommen von $a$ in $w$.
\end{definition}


\subsection{Beispiele für Kodierungen}
\subsubsection{Natürliche Zahlen}
Sei $m$ eine natürliche Zahl. Dann erzeugt die Funktion $\operatorname{Bin}(m) \in \Sigma_\text{bool}^*$ die binäre Darstellung der natürlichen Zahl $m$. Damit $\operatorname{Bin}(m)$ eindeutig ist, soll die Funktion die kürzeste binäre Darstellung liefern (das erste Zeichen ist eine 1).

Die Umkehrfunktion ist $\text{Nummer}(x) = \sum_{i = 1}^n x_i \cdot 2^{n-i}$ und erzeugt für eine binäre Darstellung $x$ die natürliche Zahl.

\subsubsection{Graphen}
Gerichtete Graphen $G = (V, E)$ ($V$ ist die Knotenmenge, $E$ die Kantenmenge) können durch eine Adjazenzmatrix $M_G$ beschrieben werden: $M_G = [a_{ij}]$. Falls der Knoten $v_i \in V$ mit dem Knoten $v_j \in V$ verbunden ist, so gilt für $M_G$: $a_{ij} = 1$, sonst $a_{ij} = 0$. Die Adjazenzmatrix kann nun durch ein Wort über dem Alphabet $\Sigma = \{0, 1, \#\}$ beschrieben werden. Dazu schreibt man den Inhalt jeder Zeile nacheinander und trennt die Zeilen im entstehenden Wort durch $\#$.

Möchte man einen gewichteten Graphen $G = (V, E, h)$ mit einer Funktion $h(e) \in \N - \{0\}$ für eine Kante $e \in E$, so kann dies ebenfalls über dem Alphabet $\Sigma = \{0, 1, \#\}$ gemacht werden. Wieder schreibt man den Inhalt jeder Zeile der Adjazenzmatrix nacheinander. Dabei kodiert man die Gewichtung mittels $\operatorname{Bin}(h(e))$, trennt die einzelnen Matrixeinträge durch $\#$ ab und Zeilen durch $\#\#$.

\subsubsection{Bool'sche Formeln}
Für Bool'sche Formeln verwenden wir das Alphabet $\Sigma_\text{logic} = \{0, 1, x, (, ), \land, \lor, \lnot\}$. In Bool'schen Formeln kommen, im Gegensatz zu unserem Alphabet, beliebig viele Variabeln $x_i$ vor. Unser Alphabet muss aber gleichzeitig endlich sein. Deshalb werden die Variabeln $x_i$ durch das Wort $x\operatorname{Bin}(i)$ kodiert. Die restlichen Symbole können direkt übernommen werden.

\subsection{Konkatenation}
\begin{definition}
Die \textbf{Verkettung (Konkatenation)} für ein Alphabet $\Sigma$ ist eine Abbildung $K: \Sigma^* \times \Sigma^* \to \Sigma^*$, so dass für alle $x, y \in \Sigma^*$
\[
K(x, z) = x \cdot y = xy
\]
gilt.\\
\end{definition}

\begin{remark}
Die Verkettung $K$ über $\Sigma$ ist eine assoziative Operation:
\[
K(u, K(v, w)) = u \cdot (v \cdot w) = u \cdot (vw) = uvw = (u \cdot v) \cdot w = K(K(u, v), w)
\]\\
\end{remark}

\begin{remark}
Für jedes $w \in \Sigma^*$ gilt
\[
w \cdot \lambda = \lambda \cdot w = x
\]\\
\end{remark}

\begin{remark}
Die Konkatenation ist nur für einelementige Alphabete kommutativ.\\
\end{remark}

\begin{remark}
Für alle $x, y \in \Sigma^*$ gilt:
\[
|xy| = |x \cdot y| = |x| + |y|
\]
\end{remark}

\begin{definition}
Sei $\Sigma$ ein Alphabet. Für alle $x \in \Sigma^*$ und alle $i \in \N$ wird die $i$-te Iteration $x^i$ von $x$ definiert als:
\[
x^0 = \lambda,\quad x^1 = x,\quad x^i = x \cdot x^{i-1}
\]
\end{definition}

\subsection{Teilworte}
\begin{definition}
Seien $u, w \in \Sigma^*$ für ein Alphabet $\Sigma$.
\begin{itemize}
  \item $v$ ist ein \textbf{Teilwort} von $w$ $\Leftrightarrow \exists x, y \in \Sigma^*: w = xvy$
  \item $v$ ist ein \textbf{Suffix} von $w$ $\Leftrightarrow \exists x \in \Sigma^*: w = xv$
  \item $v$ ist ein \textbf{Präfix} von $w$ $\Leftrightarrow \exists x \in \Sigma^*: w = vx$
  \item $v$ ist ein \textbf{echtes} Teilwort/Suffix/Präfix von $w$ genau dann, wenn $v \not= w$, $v \not= \lambda$ und $v$ ist ein Teilwort/Suffix/Präfix von $w$
\end{itemize}

\end{definition}

\subsection{Ordnung}
\begin{definition}
Sei $\Sigma = \{s_1, s_2, \ldots, s_m\}$ ein Alphabet für ein beliebiges $m \geq 1$. Weiter sei $s_1 < s_2 < s_3 < \ldots < s_m$ eine Ordnung auf $\Sigma$. Darauf basierend wir die \textbf{kanonische Ordnung} auf $\Sigma^*$ für $u, v \in \Sigma^*$ wie folgt definiert:
\begin{align*}
u < v \Leftrightarrow \quad & |u| < |v|\\
&\lor (|u| = |v| \land u = x \cdot s_i \cdot u' \land v = x \cdot s_j \cdot v'\\
&\text{für beliebige } x, u', v' \in \Sigma^* \text{und } i < j)
\end{align*}

\end{definition}

\section{Sprache}
Eine Sprache wird durch eine beliebige Menge von Wörtern über einem festen Alphabet gebildet.

\begin{definition}
Eine \textbf{Sprache} $L$ über einem Alphabet $\Sigma$ ist eine Teilmenge von $\Sigma^*$.\\
\end{definition}

\begin{itemize}
  \item $L_\emptyset = \emptyset$ ist die leere Sprache (hat keine Elemente)
  \item $L_\lambda = \{\lambda\}$ ist die einelementige Sprache, die nur das leere Wort enthält\\
\end{itemize}

\begin{definition}
Sind $L_1$ und $L_2$ zwei Sprachen über demselben Alphabet $\Sigma$, so ist
\[
L_1 \cdot L_2 = L_1 L_2 = \{vw | v \in L_1, w \in L_2\}
\]
die \textbf{Konkatenation} von $L_1$ und $L_2$.\\
\end{definition}

\begin{definition}
Ist $L$ eine Sprache über $\Sigma$, so wird definiert:
\begin{itemize}
  \item $L^0 = L_\lambda$
  \item $L^{i+1} = L^i \cdot L, \quad \forall i \in \N$
  \item $L^* = \bigcup_{i \in \N} L^i$ (ist der \textbf{Kleene'sche Stern})
  \item $L^+ = \bigcup_{i \in \N - \{0\}} L^i = L \cdot L^*$\\
\end{itemize}
\end{definition}

\begin{remark}
\
\begin{itemize}
  \item $\Sigma^i = \{w | w \in \Sigma^* \land |w| = i\}$
  \item $L_\emptyset L = L_\emptyset = \emptyset$
  \item $L_\lambda L = L$\\
\end{itemize}

\end{remark}

\begin{lemma}
Seien $L_1, L_2, L_3$ Sprachen über dem Alphabet $\Sigma$. Dann gilt $L_1 L_2 \cup L_1 L_3 = L_1 (L_2 \cup L_3)$.\\
\end{lemma}

\begin{lemma}
Seien $L_1, L_2, L_3$ Sprachen über dem Alphabet $\Sigma$. Dann gilt $L_1 (L_2 \cap L_3) \subseteq L_1 L_2 \cap L_1 L_3$\\
\end{lemma}

\begin{definition}
Seien $\Sigma_1, \Sigma_2$ zwei Alphabete. Ein \textbf{Homomorphismus} von $\Sigma_1^*$ nach $\Sigma_2^*$ ist jede Funktion $h: \Sigma_1^* \to \Sigma_2^*$ mit folgenden Eigenschaften:
\begin{itemize}
  \item $h(\lambda) = \lambda$
  \item $h(uv) = h(u) \cdot h(v) \quad \forall u,v \in \Sigma_1^*$\\
\end{itemize}
\end{definition}

\begin{remark}
Um einen Homomorphismus zu spezifizieren reicht es aus für alle Zeichen $a \in \Sigma_1$ $h(a)$ zu definieren.
\end{remark}

\section{Algorithmische Probleme}
Ein Programm ist im Grunde eine Abbildung $A$, welche ein Wort über $\Sigma_1$ in ein Wort über $\Sigma_2$ abbildet: $A: \Sigma_1^* \to \Sigma_2^*$. Somit ist sowohl die Eingabe, wie auch die Ausgabe des Programms als Wort kodiert und $A$ bestimmt für jedes Eingabewort ein bestimmtes Ausgabewort.


Zwei Programme $A$ und $B$ sind \textbf{äquivalent}, wenn für alle $x \in \Sigma_1^*$ $A(x) = B(x)$ gilt.

\subsection{Entscheidungsproblem}
\begin{definition}
Das \textbf{Entscheidungsproblem $(\Sigma, L)$} für ein gegebenes Alphabet $\Sigma$ und eine gegebene Sprache $L \subseteq \Sigma^*$ ist, für jedes $x \in \Sigma^*$ zu entscheiden ob
\[
x \in L \text{ oder } x \not\in L.
\]\\
\end{definition}

\begin{definition}
Ein Algorithmus $A$ \textbf{löst} das Entscheidungsproblem $(\Sigma, L)$, falls für alle $x \in \Sigma^*$ gilt:
\[
A(x) =
\begin{cases}
1, &\text{falls } x \in L \\
0, &\text{falls } x \not \in L
\end{cases}
\]
In diesem Fall sagt man, dass $A$ die Sprache $L$ \textbf{erkennt}.\\
\end{definition}

\begin{definition}
Wenn für eine Sprache $L$ ein Algorithmus existiert, der $L$ erkennt, so sagt man, dass $L$ \textbf{rekursiv} ist.\\
\end{definition}


\begin{definition}
Seien $\Sigma, \Gamma$ zwei Alphabete. Wir sagen, dass ein Algorithmus $A$ eine \textbf{Funktion $f: \Sigma^* \to \Gamma^*$} berechnet, falls
\[
\forall x \in \Sigma^*: A(x) = f(x)
\]
\end{definition}

Das Entscheidungsproblem ist ein Spezialfall einer Funktionsberechnung.

\subsection{Relationsproblem}
\begin{definition}
Seien $\Sigma, \Gamma$ zwei Alphabete, und sei $R \subseteq \Sigma^* \times \Gamma^*$ eine Relation. Ein Algorithmus $A$ löst das \textbf{Relationsproblem} $R$, falls für jedes $x \in \Sigma^*$ gilt:
\[
(x, A(x)) \in R
\]
\end{definition}

\subsection{Optimierungsproblem}
\begin{definition}
Ein \textbf{Optimierungsproblem} ist ein 6-Tupel $\mathcal{U} = (\Sigma_I, \Sigma_O, L, \mathcal{M}, \operatorname{cost}, \operatorname{goal})$:
\begin{itemize}
  \item $\Sigma_I$ ist das Eingabealphabet
  \item $\Sigma_O$ ist das Ausgabealphabet
  \item $L \subseteq \Sigma_I^*$ ist die Sprache der zulässigen Eingaben
  \item $\mathcal{M}$ ist eine Funktion $\mathcal{M}: L \to \mathcal{P}(\Sigma_O^*)$. Für jedes $x \in L$ ist $\mathcal{M}(x)$ die Menge der zulässigen Lösungen für $x$
  \item $\operatorname{cost}$ ist eine Funktion $\operatorname{cost}: \bigcup_{x \in L}(\mathcal{M} \times \{x\}) \to \R^+$ und ist die Preisfunktion
  \item $\operatorname{goal} \in \{\text{Minimum}, \text{Maximum}\}$ ist das Optimierungsziel
\end{itemize}

Eine zulässige Lösung $\alpha \in \mathcal{M}(x)$ heisst \textbf{optimal} für den Problemfall $x$ des Optimierungsproblems $U$, falls
\[
\operatorname{cost}(\alpha, x) = \operatorname{Opt}_\mathcal{U} = \operatorname{goal}\{\operatorname{cost}(\beta, x) | \beta \in \mathcal{M}(x)\}
\]\\
\end{definition}

\begin{definition}
Ein Algorithmus $A$ \textbf{löst} $\mathcal{U}$, falls für jedes $x \in L$:
\begin{enumerate}
  \item $A(x) \in \mathcal{M}(x)$ ($A(x)$ ist eine zulässige Lösung des Problemfalls $x$ von $\mathcal{U}$)
  \item $\operatorname{cost}(A(x), x) = \operatorname{goal}\{\operatorname{cost}(\beta, x) | \beta \in \mathcal{M}(x)\}$
\end{enumerate}

\end{definition}

\begin{remark}
Oft wird die Spezifikation von $\Sigma_I, \Sigma_O$ bei Optimierungsproblemen weggelassen. Man geht davon aus, dass die verwendeten Daten kodiert werden können für ein $\Sigma_I, \Sigma_O$. So bleiben noch vier Dinge übrig, die spezifiziert werden müssen:
\begin{enumerate}
  \item die Menge der Problemfälle $L$, also die zulässigen Eingaben
  \item die Menge der Einschränkungen, gegeben durch jeden Problemfall $x \in L$, und damit $\mathcal{M}(x)$ für jedes $x \in L$. $\mathcal{M}(x)$ gibt uns Lösungen, die den Einschränkungen genügen für ein gegebenes $x \in L$
  \item die Kostenfunktion
  \item das Optimierungsziel
\end{enumerate}

\end{remark}

\subsubsection{Beispiel: Traveling Salesman Problem (TSP)}
\begin{description}
  \item[Eingabe:] Ein gewichteter Graph $(G, c)$, wobei $G = (V, E)$ ein Graph ist und $c: E \to \N - \{0\}$ die Kostenfunktion. Strikt formal müsste man das Eingabealphabet eingeben und mit diesem den Graphen kodieren.
  \item[Einschränkungen:] Für jeden Problemfall $(G, c)$ ist $\mathcal{M}(G, c)$ die Menge aller Hamiltonscher Kreise von $G$ mit der Kostenfunktion $e$.
  \item[Kosten:] Für jeden Hamiltonschen Kreis $H = v_{i_1}, v_{i_2}, \ldots, v_{i_n}, v_{i_1} \in \mathcal{M}(G, c)$:
    \[
    \operatorname{cost}((v_{i_1}, v_{i_2}, \ldots, v_{i_n}, v_{i_1}), (G, c)) = \sum_{j = 1}^n c \left (\{v_{i_j}, v_{i_{(j \mod n) + 1}}\} \right )
    \]
    Die Kosten jedes Hamiltonschen Kreises ist somit die Summe der Gewichte der besuchten Kanten.
  \item[Ziel:] Minimum\\
\end{description}

\begin{definition}
Ein Optimierungsproblem $\mathcal{U}_1 = (\Sigma_I, \Sigma_O, L', \mathcal{M}, \operatorname{cost}, \operatorname{goal})$ ist ein \textbf{Teilproblem} vom Optimierungsproblem $\mathcal{U}_2 = (\Sigma_I, \Sigma_O, L, \mathcal{M}, \operatorname{cost}, \operatorname{goal})$, falls $L' \subseteq L$.
\end{definition}

\subsubsection{Knotenüberdeckung}
\label{sec:knotenueberdeckung}
\begin{definition}
Eine \textbf{Knotenüberdeckung} eines Graphen $G = (V, E)$ ist jede Knotenmenge $U \subseteq V$, so dass jede Kante aus $E$ mit mindestens einem Knoten aus $U$ inzident ist. Eine Kante $\{u, v\} \in E$ ist inzident zu $u$ und $v$.

Anders gesagt: Jede Kante muss an mindestens einem Ende in einem Knoten enden, der in der Knotenmenge der Knotenüberdeckung ist.
\end{definition}

\subsubsection{Beispiel: Maximale Clique Problem (MAX-CL)}
\begin{definition}
Eine Clique eines Graphen $G = (V, E)$ ist jede Teilmenge $U \subseteq V$, so dass $\{\{u, v\} | u, v \in U, u \not= v\} \subseteq E$ (die Knoten von $U$ bilden einen vollständigen Teilgraphen von $G$).
\end{definition}

Das Maximale Clique Problem besteht nun darin eine Clique mit maximaler Kardinalität zu finden. Wir suchen also einen vollständigen Teilgraphen mit maximaler Anzahl von Knoten. Ein vollständiger Teilgraph ist ein Graph $H = (U, F), U \subseteq V, F \subseteq E$, so dass $F$ alle Kanten aus $E$ enthält, die zwei Knoten in $U$ verbinden.

\begin{description}
  \item[Eingabe:] Ein (ungerichteter) Graph $G = (V, E)$
  \item[Einschränkung:] $\mathcal{M}(G) = \{S \subseteq V | \{\{u, v\} | u, v \in S, u \not= v\} \subseteq E\}$
  \item[Kosten:] Für jedes $S \in \mathcal{M}(G)$ ist $\operatorname{cost}(S, G) = |S|$
  \item[Ziel:] Maximum
\end{description}

\subsubsection{Beispiel: Maximale Erfüllbarkeit (MAX-SAT)}
Sei $X = \{x_1, x_2, \ldots \}$ die Menge der Bolle'schen Variabeln. Sei $Lit_X = \{x, \overline{x} | x \in X\}$ die Menge der Literale. Dabei ist $\overline{x}$ die negation von $x$. Eine Klausel ist eine beliebig grosse endliche Disjunktion von Literalen (z.B. $x_1 \lor \overline{x_2} \lor x_3$).

Eine Formel ist in \textbf{konjunktiver Normalform (KNF)}, falls sie eine Konjunktion von Klauseln ist. Also eine Konjunktion von Disjunktionen. Beispiel: $(x_1 \lor \overline{x_2} \lor \overline{x_3}) \land (x_4 \lor \overline{x_5})$.

Das Problem der maximalen Erfüllbarkeit ist, für eine gegebene Formel $\Phi$ in KNF, eine Belegung der Variabeln zu finden, die die maximale mögliche Anzahl an Klauseln von $\Phi$ erfüllt.

\begin{description}
  \item[Eingabe:] Eine Formel $\Phi$ in KNF
  \item[Einschränkung:] Für jede Formel $\Phi$ über $\{x_{i_1}, x_{i_2}, \ldots, x_{i_n}\}$ ist $\mathcal{M}(\Phi) = \{0, 1\}^n$. Jedes $\alpha = \alpha_1 \alpha_2 \ldots \alpha_n \in \mathcal{M}(\Phi), \alpha_j \in \{0, 1\}$ für $j = 1, 2, \ldots, n$ stellt eine Belegung für die Variable $x_{i_j}$ dar.
  \item[Kosten:] Für jedes $\Phi$ und jedes $\alpha \in \mathcal{M}(\Phi)$ ist $\operatorname{cost}(\alpha, \Phi)$ die Anzahl der Klauseln, die durch $\alpha$ erfüllt werden.
  \item[Ziel:] Maximum
\end{description}

\subsubsection{Beispiel: Ganzzahlige Lineare Programmierung (integer linear programming, ILP)}
Für ein gegebenes System von linearen Gleichungen und eine lineare Funktion von Unbekannten des linearen Systems soll eine Lösung dieses Systems berechnet werden. Die Lösung soll dabei minimal sein bezüglich der gegebenen linearen Funktion.

\begin{description}
  \item[Eingabe:] Eine $m \times n$ Matrix $A$ und zwei Vektoren $b = (b_1, \ldots, b_m)^T$ und $c = (c_1, \ldots, c_n)$, wobei die Elemente von $A, b, c$ ganze Zahlen sind.
  \item[Einschränkung:] $\mathcal{M}(A, b, c) = \{ X = (x_1, \ldots, x_n)^T \in \N^n | AX = b\}$. $\mathcal{M}(A, b, c)$ enthält somit alle Lösungsvektoren $X$ für die $AX = b$ gilt.
  \item[Kosten:] Für jedes $X = (x_1, \ldots, x_n) \in \mathcal{M}(A, b, c)$ ist $\operatorname{cost}(X, (A, b, c)) = \sum_{i=1}^n c_i x_i$
  \item[Ziel:] Minimum
\end{description}

\subsection{Weitere Algorithmische Probleme}
\begin{definition}
Sei $\Sigma$ ein Alphabet, $x \in \Sigma^*$. Ein Algorithmus $A$ \textbf{generiert} das Wort $x$, falls $A$ für die Eingabe $\lambda$ die Ausgabe $x$ liefert.\\
\end{definition}

Das folgende Programm erzeugt beispielsweise das Wort $1001$:
\begin{lstlisting}
begin
  write(1001);
end
\end{lstlisting}

Ein solches Programm kann als alternative Darstellung des Wortes verwendet werden.\\

\begin{definition}
Sei $\Sigma$ ein Alphabet und $L \subseteq \Sigma^*$ eine Sprache. Ein \textbf{Aufzählungsalgorithmus $A$ für $L$} gibt für die Eingabe $n \in \N - \{0\}$ die Wortfolge $x_1, x_2, \ldots, x_n$ aus, wobei $x_1, x_2, \ldots, x_n$ die kanonisch ersten $n$ Wörter aus $L$ sind.
\end{definition}

\section{Kolmogorov-Komplexität}
\begin{definition}
Für jedes Wort $x \in \Sigma_\text{bool}^*$ ist die \textbf{Kolmogorov-Komplexität $K(x)$} des Wortes $x$ die binäre Länge des kürzesten Pascal-Programms, das $x$ generiert.
\end{definition}

\begin{lemma}
Es existiert eine Konstante $d$, so dass für jedes $x \in \Sigma_\text{bool}^*$
\[
K(x) \leq |x| + d
\]
\end{lemma}

Die Länge eines solchen Programms ist also nicht wesentlich länger als das Wort selbst. Wir können für jedes Wort $x \in \Sigma_\text{bool}^*$ das Programm
\begin{lstlisting}
begin
  write(x);
end
\end{lstlisting}

Man sieht, dass bloss das Wort $x$ variabel im Programm ist und der Rest eine fixe Länge hat.

Das Wort $x$ wird im Programm in binärer Form eingefügt und leistet deswegen zur binären Darstellung des Programms nur den Beitrag $|x|$.\\

\begin{remark}
Für ein Wort $x \in \Sigma_\text{bool}^*$ braucht das Abspeichern von $\operatorname{Bin}(|x|)$ so viel Platz: $\lceil \log_2 (\operatorname{Bin}(|x|) + 1) \rceil$
\end{remark}

\subsection{Beispiel \#1}
Um die Wörter der Form $y_n = 0^n \in \{0, 1\}^*$ für jedes $n \in \N - \{0\}$ zu generieren, könnte das folgende Programm eingesetzt werden:
\begin{lstlisting}
begin
  for l = 1 to |$n$| do
    write(0);
end
\end{lstlisting}

Wobei sich je nach Wort nur die Kodierung von $n$ im Programm verändert. Somit erhalten wir für die Länge der Kodierung von $n$: $\lceil \log_2(n + 1) \rceil$. Dies gibt uns eine Kolmogorov-Komplexität von $K(y_n) \leq \lceil \log_2(n+1) \rceil + c = \lceil \log_2 |y_n| \rceil + c$

\subsection{Beispiel \#2}
Das Wort $z_n = 0^{n^2} \in \{0, 1\}^*$ für jedes $n \in \N - \{0\}$ kann durch folgendes Programm dargestellt werden:
\begin{lstlisting}
begin
  M := n;
  M := M |$\times$| M;
  for l = 1 to M do
    write(0);
end
\end{lstlisting}

Hier ist wieder bloss $n$ abhängig vom Wort, der Rest des Programms hat eine feste Länge. Wir erhalten somit
\[
K(z_n) \leq \lceil \log_2 (n+1) \rceil + d \leq \lceil \log_2 (\sqrt{|z_n|}) \rceil + d + 1
\]

\subsection{Weiteres}
\begin{definition}
Die Kolmogorov-Komplexität einer natürlichen Zahl $n$ ist $K(n) = K(\operatorname{Bin(n)})$.
\end{definition}

\begin{lemma}
Für jede Zahl $n \in \N - \{0\}$ existiert ein Wort $w_n \in (\Sigma_\text{bool}^n$, so dass $K(w_n) \geq |w_n| = n$, es existiert somit für jede Zahl $n$ ein nichtkomprimierbares Wort der Länge $n$.\\
\end{lemma}

\begin{satz}
Seien $A$ und $B$ zwei Programmiersprachen. Dann existiert die Konstante $c_{A, B}$, die nur von $A$ und $B$ abhängig ist, so dass:
\[
|K_A(x) - K_B(x)| \leq c_{A, B} \quad \forall x \in \Sigma_\text{bool}^*
\]

Daraus folgt, dass die verwendete Programmiersprache nicht relevant ist für die Berechnung der Kolmogorov-Komplexität.\\
\end{satz}

\begin{definition}
Ein Wort $x \in \Sigma_\text{bool}^*$ ist \textbf{zufällig}, falls $K(x) \geq |x|$.

Eine Zahl $n \in \N$ ist zufällig, falls $K(n) = K(\operatorname{Bin}(n)) \geq \lceil \log_2(n+1) \rceil -1$.
\end{definition}

\todo[inline]{Satz 2.2, Satz 2.3 (Primzahlsatz)}
\chapter{Endliche Automaten}
\begin{itemize}
  \item Endliche Automaten sind das einfachste Berechnungsmodell, welches in der Informatik betrachtet wird.
  \item Sie entsprechen speziellen Programmen, die Entscheidungsprobleme lösen.
  \item Endliche Automaten verwenden dabei keine Variabeln.
  \item Die Eingabe wird nur einmal von links nach rechts gelesen.
  \item Nach dem Lesen des letzten Buchstabens steht das Resultat sofort fest.
\end{itemize}

\section{Darstellung endlicher Automaten}
\begin{definition}
Ein (deterministischer) \textbf{endlicher Automat (EA)} ist ein Quintupel $M = (Q, \Sigma, \delta, q_0, F)$:
\begin{itemize}
  \item $Q$ ist eine endliche Menge von \textbf{Zuständen}
  \item $\Sigma$ ist ein Alphabet, welches als \textbf{Eingabealphabet} bezeichnet wird
  \item $q_0 \in Q$ ist der \textbf{Anfangszustand}
  \item $F \subseteq Q$ ist die \textbf{Menge der akzeptierenden Zustände}
  \item $\delta$ eine Funktion $\delta: Q \times \Sigma \to Q$, welche als \textbf{Übergangsfunktion} bezeichnet wird. Daher bedeutet $\delta(q_i, a) = p$, dass falls $M$ im Zustand $q_i \in Q$ den Buchstaben $a \in \Sigma$ liest, es in den Zustand $p \in Q$ übergeht.\\
\end{itemize}

\end{definition}

\begin{definition}
Eine \textbf{Konfiguration} von $M$ ist ein Element aus $Q \times \Sigma^*$. Falls $M$ in der Konfiguration $(q_i, w) \in Q \times \Sigma^*$ ist, so bedeutet es, dass $M$ aktuell im Zustand $q_i \in Q$ ist und noch den Suffix $w \in \Sigma^*$ des Eingabeworts zu lesen hat.

Die Konfiguration $(q_0, x) \in \{q_0\} \times \Sigma^*$ nennt man eine \textbf{Startkonfiguration} von $M$ auf $x$. Die Berechnung des Worts $x$ beginnt somit an der Startkonfiguration $q_0$.\\
\end{definition}

\begin{definition}
Eine \textbf{Endkonfiguration} von $M$ hat die Form $(q_i, \lambda) \in Q \times \{ \lambda \}$\\
\end{definition}

\begin{definition}
Ein \textbf{Schritt} von $M$ ist eine Relation $\vdash_M \subseteq (Q \times \Sigma^*) \times (Q \times \Sigma^*)$ und ist definiert durch
\[
(q, w) \vdash_{M} (p, x) \Leftrightarrow w = ax,\ a \in \Sigma \land \delta(q, a) = p
\]

Es handelt sich somit um eine Relation auf Konfigurationen des EA $M$. Es beschreibt den Übergang von einer Konfiguration in die nächste, nachdem der nächste Buchstabe von $w$ (hier der Buchstabe $a$) gelesen wurde.\\
\end{definition}

\begin{definition}
Eine \textbf{Berechnung} $C$ von $M$ ist eine endliche Folge $C = C_0, C_1, C_2, C_3, \ldots, C_n$ von Konfigurationen ($C_i \in (Q \times \Sigma^*),\ i=0 \ldots n$), so dass $C_i \vdash_M C_{i+1}$ gilt für alle $0 \leq i \leq n-1$.

Falls $C_0 = (q_0, x)$ und $C_n \in Q \times \{ \lambda \}$, so ist $C$  die Berechnung von $M$ auf einer Eingabe $x \in \Sigma^*$.

Falls $C_n \in F \times \{ \lambda \}$, so sagen wir, dass $C$ eine \textbf{akzeptierende Berechnung} von $M$ auf $x$ ist, und dass $M$ das Wort $x$ akzeptiert.

Falls $C_n \in (Q - F) \times \{ \lambda \}$, so ist $C$ eine \textbf{verwerfende Berechnung} von $M$ auf $x$, und dass $M$ das Wort $x$ verwirft.\\
\end{definition}

\begin{definition}
Die von $M$ akzeptierte Sprache $L(M)$ ist definiert als
\begin{align*}
L(M) := \{ w \in \Sigma^* | \text{ die Berechnung von $M$ auf $w$ endet in einer}\\ \text{Endkonfiguration $(q, \lambda)$ mit $q \in F$} \}
\end{align*}
\end{definition}

\begin{definition}
$\mathcal{L}(EA) = \{L(M) | M \text{ ist ein EA} \}$ ist die Klasse der Sprachen, die von endlichen Automaten akzeptiert werden. $\mathcal{L}(EA)$ bezeichnet man auch als die \textbf{Klasse der regulären Sprachen}, und jede Sprache $L$ aus $\mathcal{L}(EA)$ als \textbf{regulär}.\\
\end{definition}

\begin{definition}
Sei $M$ ein EA. Wir definieren $\vdash_M^*$ als die reflexive und transitive Hülle der Schrittrelation $\vdash_M$ von $M$. Somit gilt
\[
(q, w) \vdash_M^* (p, u) \Leftrightarrow (q = p \land w = u) \lor \exists k \in \N - \{0\} \text{ so dass}
\]
\begin{itemize}
  \item $w = a_1 a_2 a_3 a_4 \ldots a_k u, \ a_i \in \Sigma \ i = 1, \ldots, k$ und
  \item $\exists r_1, r_2, \ldots, r_{k-1} \in Q$, so dass
  $(q, w) \vdash_M (r_1, a_2\ldots a_k u) \vdash_M (r_2, a_3 \ldots a_k u) \vdash_M \ldots \vdash_M (r_{k-1}, a_k u) \vdash_M (p, u)$
\end{itemize}

$(q, w) \vdash_M^* (p, u)$ sagt also aus, dass es eine Berechnung von $M$ gibt, die ausgehen von der Konfiguration $(q, w)$ zur Konfiguration $(p, u)$ führt.\\
\end{definition}

\begin{definition}
Sei $\hat{\delta}: Q \times \Sigma^* \to Q$ definiert durch:
\begin{itemize}
  \item $\hat\delta(q, \lambda) = q$ für alle $q \in Q$
  \item $\hat\delta(q, wa) = \delta(\hat\delta(q, w), a)$ für alle $a \in \Sigma,\ w \in \Sigma^*,\ q \in Q$
\end{itemize}

Wenn $M$ im Zustand $q$ ist und das Wort $w$ zu lesen beginnt, dann bedeutet $\hat\delta(q, w) = p$ dass $M$ im Zustand $p$ enden wird. Oder anders gesagt, es gilt: $(q, w) \vdash_M^* (p, \lambda)$.

Gekürzt können wir nun $L(M)$ wie folgt definieren:
\[
L(M) = \{ w \in \Sigma^* \ |\ (q_0, w) \vdash_M^* (p, \lambda),\ p \in F \} = \{ w \in \Sigma^* \ |\ \hat\delta(q_0, w) \in F \}
\]\\
\end{definition}

\section{Simulation}
\begin{lemma}
Sei $\Sigma$ ein Alphabet und $M_1 = (Q_1, \Sigma, \delta_1, q_{01}, F_1), M_2 = (Q_2, \Sigma, \delta_2, q_{02}, F_2)$ zwei EA. Für $\circledcirc \in \{\cup, \cap, -\}$ existiert jeweils ein EA $M$ mit $L(M) = L(M_1) \circledcirc L(M_2)$.
\end{lemma}

\begin{proof}
Die Existenz von $L(M)$ basiert auf der Idee einer Konstruktion von $M$ in welchem $M_1$ und $M_2$ simuliert werden. Dabei sind die Zustände von $M$ Paare der Form $(q, p) \in Q_1 \times Q_2$.

Formale konstruktion von $M$: Sei $M = (Q, \Sigma, \delta, q_0, F_\circledcirc)$ und:
\begin{itemize}
  \item $Q = Q_1 \times Q_2$
  \item $q_0 = (q_{01}, q_{02})$
  \item $\forall q \in Q_1, p \in Q_2, a \in \Sigma:\, \delta((q, p), a) = (\delta_1(q, a), \delta_2(p, a))$
  \item F ist:
  \begin{itemize}
    \item Falls $\circledcirc = \cup$, dann ist $F = F_1 \times Q_2 \cup Q_1 \times F_2$
    \item Falls $\circledcirc = \cap$, dann ist $F = F_1 \times F_2$
    \item Falls $\circledcirc = -$, dann ist $F = F_1 \times (Q_2 - F_2)$
  \end{itemize}
\end{itemize}

Um nun zu beweisen, dass ein solcher EA $M$ für die Operation $\circledcirc$ existiert, reicht es zu zeigen dass folgende Gleichheit gilt: $\forall x \in \Sigma^*: \hat\delta((q_{01}, q_{02}), x) = (\hat\delta_1(q_{01}, x), \hat\delta_2(q_{02}, x))$. Der Beweis kann mittels Induktion über der Länge von $x$ geführt werden.
\end{proof}

Diese Entwurfsmethode kann dazu eingesetzt werden endliche Automaten für komplexere Sprachen zu bauen, in dem EA für einfachere Sprachen zusammengesetzt werden. So kann die Sprache $L = \{x \in \Sigma_\text{bool}^* |\ |x|_0 \mod 3 = 1, |x|_1 \mod 3 = 2\}$ aus den Sprachen $L_1 = \{x \in \Sigma_\text{bool}^* |\ |x|_0 \mod 3 = 1\}$ und $L_2 = \{x \in \Sigma_\text{bool}^* |\ |x|_1 \mod 3 = 2\}$ konstruiert werden.

\section{Beweise der Nichtexistenz}
\subsection{Lemma 3.3}
\begin{lemma}[Lemma 3.3]
Sei $A = (Q, \Sigma, \delta_A, q_0, F)$ ein EA. Seien weiter $x, y \in \Sigma^*, x \not= y$, so dass $(q_0, x) \vdash_A^* (p, \lambda) \land (q_0, y) \vdash_A^* (p, \lambda)$ für ein $p \in Q$. Dann gilt für jedes $z \in \Sigma^*:\, xz \in L(A) \Leftrightarrow yz \in L(A)$.
\end{lemma}

Anders ausgedrückt: Wenn wir zwei Wörter betrachten, die im EA $A$ im selben Zustand landen, so haben wir ab diesem Moment keine Information mehr, welches der beiden Wörter wir gelesen haben. Konkatinieren wir nun ein Wort $z \in \Sigma^*$ zu den beiden ausgewählten Wörtern, so müssen beide Wörter in $L(A)$ sein oder beide nicht.

Dieses Lemma kann dazu eingesetzt werden zu zeigen, dass eine Sprache nicht regulär ist ($L \not\in \mathcal{L}(EA)$). Dazu führt man den Beweis indirekt und nimmt an, dass $L$ regulär sei und wendet das Lemma an. Dabei wird festgestellt, dass es ein $z$ gibt, für welches $xz \in L$, aber $yz \not\in L$, was ein Wiederspruch ist.

\subsubsection{Beispiel}
Es sei zu zeigen, dass $L = \{0^n1^n | n \in \N\}$ nicht regulär ist.
\begin{proof}
Wir führen den Beweis indirekt mittels dem Lemma 3.3. Sei $L \in \mathcal{L}(EA)$. Somit existiert ein EA $A = (Q, \Sigma, \delta, q_0, F)$ mit $L(A) = L$.

Wir betrachten die Wörter $0^1, 0^2, \ldots, 0^{|Q|+1}$. Wir haben somit $|Q| + 1$ Wörter. Somit existieren $i, j \in \{1, 2, \ldots |Q| + 1\}, i < j$, so dass $\hat\delta(q_0, 0^i) = \hat\delta(q_0, 0^j)$. Nach Lemma 3.3 hat nun zu gelten: $\forall z \in \Sigma^*: 0^i z \in L \Leftrightarrow 0^j z \in L$. Dem ist aber nicht so. Für $z = 1^i$ erhalten wir: $0^i 1^i \in L$, aber auch $0^j 1^i \not\in L$ ($i < j$).

Somit ist $L \not\in \mathcal{L}(EA)$.
\end{proof}

\subsection{Pumping-Lemma}
\begin{lemma}[Pumping-Lemma für reguläre Sprachen]
Sei $L$ regulär. Dann existiert die Konstante $n_0 \in \N$, so dass jedes Wort $w \in \Sigma^*$ mit $|w| \geq n_0$ zerlegt werden kann in $w = y x z$, mit
\begin{itemize}
  \item $|yx| \leq n_0$,
  \item $|x| \geq 1$ und
  \item entweder $\{xy^kz | k \in \N\} \subseteq L$ oder $\{xy^kz | k \in \N \} \cap L = \emptyset$.
\end{itemize}

\end{lemma}

\subsubsection{Beispiel}
Es sei zu zeigen, dass $L = \{0^n 1^n | n \in \N\}$ nicht regulär ist.

\begin{proof}
Wir führen den Beweis indirekt mittels des Pumping-Lemmas. Sei $L \in \mathcal{L}(EA)$. Dann existiert eine Konstante $n_0$ mit den im Pumping-Lemma beschriebenen Eigenschaften. Wir betrachten das Wort $w = 0^{n_0} 1^{n_0}$. Es gilt offensichtlich, dass $|w| \geq n_0$. Somit kann $w$ zerlegt werden in $w = y x z$.

Bedingt durch die Eigenschaft $|yx| \leq n_0$ gilt hier, dass $y = 0^l$ und $x = 0^m$ für $l, m \in \N$. Bedingt durch die zweite Eigenschaft $|x| \geq 1$ ist $m \not= 0$. Nun prüfen wir ob mit diesen Eigenschaften, auch die dritte Pumping-Lemma Eigenschaft gilt. Wir erhalten also $\{ y x^k z | z \in \N\} = \{ 0^{n_0 - l} (0^m)^k 1^{n_0} | k \in \N \} = \{ 0^{n_0 - m + km} 1^{n_0} | k \in \N \}$. Wählen wir $k = 0$, so erhalten wir das Wort $0^{n_0 - m} 1^{n_0}$, was nicht in $L$ liegt, da $m \not= 0$. Für $k = 1$ hingegen liegt das Wort in $L$. Dies ist ein Wiederspruch. Somit ist $L$ nicht regulär.
\end{proof}

\subsection{Kolmogorov-Komplexität}
\todo[inline]{... to add ...}

\section{Nichtdeterminismus}
Determinismus bei einem deterministischen endlichen Automaten besagt, dass in jeder Konfiguration des EA eindeutig festgelegt ist, was im nächsten Schritt passiert. Somit bestimmt ein EA und ein Wort $x$ eindeutig die Berechnung von $x$ auf dem EA. Beim Nichtdeterminismus hingegen ist es erlaubt bei einer Konfiguration eine Auswahl an möglichen weiteren Schritten zu haben.

\begin{definition}
Ein \textbf{nichtdeterministischer endlicher Automat (NEA)} ist ein Quintupel $M = (Q, \Sigma, \delta, q_0, F)$ mit
\begin{itemize}
  \item $Q$ eine endliche Menge von Zuständen
  \item $\Sigma$ ist das Eingabealphabet
  \item $q_0 \in Q$ der Anfangszustand,
  \item $F \subseteq Q$ die Menge der akzeptierenden Zustände
  \item $\delta$ eine Funktion $\delta: Q \times \Sigma \to \mathcal{P}(Q)$. Die Übergangsfunktion.\\
\end{itemize}
\end{definition}

Man bemerkt, dass $Q, \Sigma, q_0, F$ die gleiche Bedeutungen haben bei einem EA wie auch bei einem NEA.\\

Der Unterschied liegt also in der $\delta$-Funktion, welche als Wertebereich nicht $Q$, sondern $\mathcal{P}(Q)$ hat. Darin liegt auch der zentrale Unterschied. Ein Übergang kann in beliebig vielen Zuständen enden, oder keinem.

Ein NEA akzeptiert ein Wort, falls es eine Berechnung gibt, die in einem Zustand $p \in F$ landet. Es ist dabei egal, wo alle alternativen Berechnungen enden. Es reicht, wenn eine in einem akzeptierenden Zustand landet.\\

\begin{definition}
Zur Übergangsfunktion $\delta$ wird $\hat\delta: Q \times \Sigma^* \to \mathcal{P}(Q)$ wie folgt definiert:
\begin{itemize}
  \item $\hat\delta(q, \lambda) = \{q\} \forall q \in Q$
  \item $\hat\delta(q, wa) = \{p | \text{ es existiert ein } r \in \hat\delta(q, w), \text{ so dass } p \in \delta(r, a)\} = \bigcup_{r \in \hat\delta(q, w)} \delta(r, a)$\\
\end{itemize}

\end{definition}

Aus der Definition sieht man, dass $\delta(q, w)$ jeweils die Menge aller Zustände aus $Q$ liefert, die aus $q \in Q$ durch das vollständige Lesen von $w$ erreichbar sind. Somit erhalten wir für einen NEA $M$: $L(M) = \{ w \in \Sigma^* |\ \hat\delta(q_0, w) \cap F \not= \emptyset \}$.\\

\begin{satz}
Zu jedem NEA \( M \) existiert ein EA \( A \), so dass \(L(M) = L(A)\).
\end{satz}

\begin{proof}
Sei \( M = (Q, \Sigma, \delta_M, q_0, F) \) ein NEA und \( A = (Q_A, \Sigma_A, \delta_A, q_{0A}, F_A) \) ein EA. \(A\) können wir dabei aus \(M\) wie folgt konstruieren:
\begin{itemize}
  \item \(Q_A = \{ \langle P \rangle \ |\ P \subset Q \}\)
  \item \(\Sigma_A = \Sigma\)
  \item \(q_{0A} = \langle \{q_0\} \rangle\)
  \item \(F_A = \{ \langle P \rangle \ |\ P \subseteq Q, P \cap F \not= \emptyset \}\)
  \item \( \delta_A: Q_A \times \Sigma_A \to Q_A \) und definiert wie folgt: \[ \forall \langle P \rangle \in Q_A, \forall a \in \Sigma_A: \delta_A(\langle P \rangle, a) = \left\langle \bigcup_{p \in P} \delta_M (p, a) \right\rangle
  \]\\
\end{itemize}

\end{proof}
\chapter{Grammatiken und Chomsky-Hierarchie}
Buchkapitel 10.1 bis und mit 10.3 sind Teil des ersten Selbststudiums

\section{Einleitung}
Endliche Automaten (EA) und Turingmaschinen (TM) erlauben es unendliche Objekte wie Sprachen und Mengen in endlicher Form zu beschreiben. Grammatiken sind eine weitere Möglichkeit Sprachen in endlicher und eindeutiger Weise zu beschreiben.

\section{Grammatiken}

\todo[inline]{Beispiel 10.1?}

\begin{definition}
Eine Grammatik $G$ ist ein 4-Tupel $G = (\Sigma_N, \Sigma_T, P, S)$ wobei die Bedeutung folgende ist:
\begin{itemize}
  \item Das Alphabet $\Sigma_N$ ist das \textbf{Nichtterminalalphabet}. Die Symbole aus $\Sigma_N$ nennt man \textbf{Nichtterminale}.
  \item Das Alphabet $\Sigma_T$ ist das \textbf{Terminalalphabet}. Die Symbole aus $\Sigma_T$ nennt man \textbf{Terminalsymbole}.
  \item $S \in \Sigma_N$ und ist das \textbf{Startsymbol}. Somit muss die Generierung eines Wortes jeweils mit dem Wort $w = S$ beginnen.
  \item $P$ ist eine endliche Teilmenge von $\Sigma^* \Sigma_N \Sigma^* \times \Sigma^*$ wobei $\Sigma = \Sigma_N \cup \Sigma_T$ ist. Die Elemente von $P$ heissen \textbf{Regeln}. Statt $(\alpha, \beta) \in P$ zu schreiben, wird meist $\alpha \to_G \beta$ geschrieben. Wobei die Bedeutung wie folgt ist: $\alpha$ kann in $G$ durch $\beta$ ersetzt werden.\\
\end{itemize}
\end{definition}

\begin{remark}
In dieser Vorlesung gilt:
\begin{itemize}
  \item Kleinbuchstaben $a, b, c, d, e$ und Ziffern werden für Terminalsymbole verwendet.
  \item Grossbuchstaben $A, B, C, D, X, Y, Z$ werden für Nichtterminale verwendet.
  \item Mit Kleinbuchstaben $u, v, w, x, y, z$ werden Wörter über $\Sigma_T$ bezeichnet.
  \item Mit griechischen Kleinbuchstaben ($\alpha, \beta, \gamma, \ldots$) werden beliebige Wörter über $\Sigma = \Sigma_T \cup \Sigma_N$ bezeichnet.\\
\end{itemize}
\end{remark}

\begin{remark}
Folgende Dinge sind zu beachten:
\begin{itemize}
  \item Es hat jeweils $\Sigma_N \cap \Sigma_T = \emptyset$ zu gelten.
  \item Durch die Anforderung $\alpha \in \Sigma^* \Sigma_N \Sigma^*$ muss $\alpha$ mindestens ein Nichtterminal enthalten.\\
\end{itemize}
\end{remark}

\begin{definition}
Sei $\gamma, \delta \in (\Sigma_N \cup \Sigma_T)^*$. $\delta$ ist aus $\gamma$ in einem Ableitungsschritt in $G$ ableitbar, $\gamma \Rightarrow_G \delta$, genau dann, wenn $\omega_1, \omega_2 \in (\Sigma_N \cup \Sigma_T)^*$ und eine Regel $(\alpha, \beta) \in P$ existieren, so dass gilt: $\gamma = \omega_1 \alpha \omega_2$ und $\delta = \omega_1 \beta \omega_2$.\\

$\delta$ ist aus $\gamma$ ableitbar in $G$, $\gamma \Rightarrow_G^* \delta$, genau dann wenn
\begin{itemize}
  \item entweder $\gamma = \delta$,
  \item oder für ein $n \in \N - \{0\}$ und $n+1$ Wörter $\omega_1, \omega_2, \ldots, \omega_n \in (\Sigma_N \cup \Sigma_T)^*$ existieren, so dass $\gamma = \omega_0, \delta = \omega_n$ und $\omega_i \Rightarrow_G \omega_{i+1}$ für $i = 0, 1, 2, \ldots, n - 1$.
\end{itemize}
In anderen Worten: Falls $\gamma \Rightarrow_G^* \delta$ gilt, so gibt es eine Folge von Ableitungsschritten, die bei $\gamma = \omega_1$ beginnt und bei $\delta = \omega_n$ endet.

Somit ist $\Rightarrow_G^*$ die reflexive und transitive Hülle von $\Rightarrow_G$.\\
\end{definition}

\begin{definition}
Falls $\omega \in \Sigma_T^*$ und $S \Rightarrow_G^* \omega$ gilt, dann sagt man, dass $\omega$ von $G$ erzeugt wird. Die von $G$ erzeugte Sprache ist somit $L(G) = \{\omega \in \Sigma_T^* | S \Rightarrow_G^* \omega\}$.
\end{definition}

\begin{remark}
Sei $\alpha \Rightarrow_G^i \beta$ eine Ableitung im Sinne von $\alpha \Rightarrow_G^* \beta$, die aus genau $i$ Schritten besteht.
\end{remark}

Grammatiken sind nichtdeterministische Erzeugungsmechanismen für Sprachen. Dies kommt daher, dass es mehrere gleiche linke Seiten geben darf und die Wahl der Anwendung einer dieser nicht festgelegt ist.

\todo[inline]{Beweistypen einbauen}

\section{Chomsky-Hierarchie}
\begin{definition}
Sei $G = (\Sigma_N, \Sigma_T, P, S)$ eine Grammatik
\begin{itemize}
  \item $G$ ist eine \textbf{Typ-0-Grammatik}. Die Typ-0-Grammatik ist die Klasse aller uneingeschränkten Grammatiken.
  \item $G$ ist \textbf{kontextsensitiv} oder \textbf{Typ-1-Grammatik}, falls $\forall (\alpha, \beta) \in P: \, |\alpha| \leq |\beta|$ gilt. Es gibt somit nicht die Möglichkeit ein Teilwort $\alpha$ durch ein kürzeres Teilwort $\beta$ zu ersetzen.
  \item $G$ ist \textbf{kontextfrei} oder \textbf{Typ-2-Grammatik}, falls $\forall (\alpha, \beta) \in P: \, \alpha \in \Sigma_N \land \beta \in (\Sigma_N \cup \Sigma_T)^*$ gilt. Alle Regeln haben also die Grundform $X \to \beta$ für ein Nichtterminal $X \in \Sigma_N$.
  \item $G$ ist \textbf{regulär} oder \textbf{Typ-3-Grammatik}, falls $\forall (\alpha, \beta) \in P: \, \alpha \in \Sigma_N \land \beta \in (\Sigma_T^* \cdot \Sigma_N \cup \Sigma_T^*)$. Somit haben alle Regeln einer regulären Grammatik entweder die Form $X \to u$ oder $X \to uY$ für $u \in \Sigma_T^*$ und $X, Y \in \Sigma_N$. Auch ist anzumerken, dass das Nichtterminal immer ganz rechts auf der rechten Seite zu stehen hat.\\
\end{itemize}
\end{definition}

\begin{remark}
Eine Sprache ist vom Typ $i$ ($i = 0, 1, 2, 3$), falls sie durch eine Typ-$i$-Grammatik erzeugt werden kann.\\
\end{remark}

Kontextfreie Sprachen haben die Eigenschaft, dass Nichtterminale unabhängig von den benachbarten Symbolen ersetzt werden können.

\section{Reguläre Grammatiken und endliche Automaten}

\begin{lemma}
$\mathcal{L}_3$ enthält alle endlichen Sprachen.\\
\end{lemma}

\begin{lemma}
$\mathcal{L}_3$ ist abgeschlossen bezüglich der Vereinigung. Somit gilt für alle Sprachen $L_1, L_2 \in \mathcal{L}_3: \, L_1 \cup L_2 \in \mathcal{L}_3$.\\
\end{lemma}

\begin{lemma}
$\mathcal{L}_3$ ist abgeschlossen bezüglich der Konkatenation. Somit gilt für alle Sprachen $L_1, L_2 \in \mathcal{L}_3: \, L_1 \cdot L_2 \in \mathcal{L}_3$.\\
\end{lemma}

\begin{satz}
Zu jedem endlichen Automaten (EA) $A$ existiert eine reguläre Grammatik $G$ mit $L(A) = L(G)$.\\
\end{satz}

\begin{definition}
Eine reguläre Grammatik (Typ-3-Grammatik) $G = (\Sigma_N, \Sigma_T, P, S)$ heisst \textbf{normiert}, wenn alle Regeln der Grammatik nur eine der folgenden drei Formen haben:
\begin{itemize}
  \item $S \to \lambda$, wobei $S$ das Startsymbol ist
  \item $A \to a$, wobei $A \in \Sigma_N$ und $a \in \Sigma_T$
  \item $B \to bC$, wobei $B, C \in \Sigma_N$ und $b \in \Sigma_T$\\
\end{itemize}
\end{definition}

\begin{lemma}
Für jede reguläre Grammatik $G$ existiert eine äquivalente \textbf{normierte} reguläre Grammatik $G'$.\\
\end{lemma}

Eine nicht normierte reguläre Grammatik $G$ kann dabei wie folgt in eine äquivalente und normierte reguläre Grammatik $G'$ überführt werden:
\begin{itemize}
  \item \textbf{Kettenregeln:} Regeln der Form $X \to Y, \, X, Y \in \Sigma_N$ enden nach endlich vielen Ableitungsschritten in $\alpha \in (\Sigma_T^* \cup \Sigma_T^+ \cdot \Sigma_N)$. Somit ersetzen wir $X \to Y$ durch $X \to \alpha$.
  \item Alle Regeln der Form $A \to \lambda$ für $A \in \Sigma_N - \{S\}$. In einer normierten regulären Grammatik darf nur aus dem Startsymbol das leere Wort abgeleitet werden. Dazu betrachten wir die Regeln $B \to \omega A, \, A \to \lambda$ womit durch das hinzufügen der Regel $B \to \omega$ und entfernen der Regel $A \to \lambda$ das Problem gelöst wird.\\
\end{itemize}

\begin{satz}
$\mathcal{L}_3 = \mathcal{L}(EA)$
\end{satz}



\chapter{Weitere Informationen}
\section{Hamiltonischer Kreis}
Ein \textbf{Hamiltonischer Kreis} eines Graphen $G$ ist ein geschlossener Weg, der jeden Knoten von $G$ genau einmal enthält.

\section{Traveling Salesman Problem (TSP)}
Fragestellung: Gegeben ist eine Menge von Städten und der Reisedistanz zwischen diesen Städten. Gesucht ist die kürzeste Route mit welcher alle Städte genau einmal besucht werden und mit der man wieder beim Anfang landet.

Es handelt sich um ein NP-vollständiges Problem.

Zur Lösung dieses Problems wird der kürzeste Hamiltonsche Kreis gesucht, der am gegebenen Startpunkt beginnt.

\section{Minimum Vertex Cover Problem (MIN-VCP)}
Gesucht wird die minimale Knotenmenge (\ref{sec:knotenueberdeckung}), die eine Knotenüberdeckung eines Graphen ist.

\section{Sprachklassen}
\subsection{Klassenübersicht}
\todo[inline]{Übersicht und Bedeutung}

\subsection{Sprachenzugehörigkeit}
\todo[inline]{Übersicht welche Sprachen zu welcher Klasse gehören}

\section{Formeln}
\subsection{Summenformel für Anzahl Programme}
\[
\sum_{i = k}^n = 2^{n+1} - 2^k
\]
\chapter{Vorlesung}
\section{Vorlesung vom 17.09.2013}
\begin{itemize}
  \item Zwei Prüfungen unter dem Semester: Mitte Semester und zweite Woche vor Semesterende. Sessionsprüfung muss man so oder so gehen. Durchschnitt der Semesterprüfungen sind die minimale Abschlussnote. Sessionsprüfung kann diese Note nur verbessern. Für Teilnahme an der Semesterprüfungen muss man den Grossteil der Serien gelöst haben
  \item Einführung in die Geschichte der Mathematik und die Entstehung der Informatik. Was unterscheidet die Informatik von anderen Wissenschaften?
  \item Alphabet, Wort und Wortlänge eingeführt
  \item Buchthemen übersprungen: Kodierung von Zahlen, Graphen und anderen Dingen als Wort
  \item Konkatenation eingeführt
\end{itemize}


\end{document}
