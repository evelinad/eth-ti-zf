\chapter{Vorlesung}
\section{Vorlesung vom 17.09.2013}
\begin{itemize}
  \item Zwei Prüfungen unter dem Semester: Mitte Semester und zweite Woche vor Semesterende. Sessionsprüfung muss man so oder so gehen. Durchschnitt der Semesterprüfungen sind die minimale Abschlussnote. Sessionsprüfung kann diese Note nur verbessern. Für Teilnahme an der Semesterprüfungen muss man den Grossteil der Serien gelöst haben
  \item Einführung in die Geschichte der Mathematik und die Entstehung der Informatik. Was unterscheidet die Informatik von anderen Wissenschaften?
  \item Alphabet, Wort und Wortlänge eingeführt
  \item Buchthemen übersprungen: Kodierung von Zahlen, Graphen und anderen Dingen als Wort
  \item Konkatenation eingeführt
\end{itemize}

\section{Vorlesung vom 20.09.2013}
\begin{itemize}
  \item $(\Sigma^*, \cdot)$ ist ein Monoid
  \item Teilwörter eingeführt
  \item Beispiel: gegeben ein Wort der Länge $k$, wie viele unterschiedliche Teilwörter gibt es?
  \item $w \in \Sigma^*, |w|_a$ eingeführt (Anzahl Vorkommen von $a$ in $w$)
  \item Kanonische Ordnung nochmals kurz repetiert
  \item Sprachen eingeführt
  \item Potenznotation bei Buchstaben eingeführt: $x^2 = xx = x \cdot x$
  \item Konkatenation von Sprachen eingeführt
  \item Potenzennotation bei Sprachen eingeführt
  \item Kleensche Stern eingeführt
  \item Mengenoperationen auf Sprachen
  \begin{itemize}
    \item Beweis für $L_1 L_2 \cup L_1 L_3 = L_1 (L_2 \cup L_3)$
    \item $L_1 L_2 \cap L_1 L_3 \not= L_1 (L_2 \cap L_3)$ erklärt
  \end{itemize}
  \item Homomorphismus eingeführt
  \item Mit Einführung in Algorithmische Probleme begonnen
\end{itemize}
