\chapter{Endliche Automaten}
\begin{itemize}
  \item Endliche Automaten sind das einfachste Berechnungsmodell, welches in der Informatik betrachtet wird.
  \item Sie entsprechen speziellen Programmen, die Entscheidungsprobleme lösen.
  \item Endliche Automaten verwenden dabei keine Variabeln.
  \item Die Eingabe wird nur einmal von links nach rechts gelesen.
  \item Nach dem Lesen des letzten Buchstabens steht das Resultat sofort fest.
\end{itemize}

\section{Darstellung endlicher Automaten}
\begin{definition}
Ein (deterministischer) \textbf{endlicher Automat (EA)} ist ein Quintupel $M = (Q, \Sigma, \delta, q_0, F)$:
\begin{itemize}
  \item $Q$ ist eine endliche Menge von \textbf{Zuständen}
  \item $\Sigma$ ist ein Alphabet, welches als \textbf{Eingabealphabet} bezeichnet wird
  \item $q_0 \in Q$ ist der \textbf{Anfangszustand}
  \item $F \subseteq Q$ ist die \textbf{Menge der akzeptierenden Zustände}
  \item $\delta$ eine Funktion $\delta: Q \times \Sigma \to Q$, welche als \textbf{Übergangsfunktion} bezeichnet wird. Daher bedeutet $\delta(q_i, a) = p$, dass falls $M$ im Zustand $q_i \in Q$ den Buchstaben $a \in \Sigma$ liest, es in den Zustand $p \in Q$ übergeht.\\
\end{itemize}

\end{definition}

\begin{definition}
Eine \textbf{Konfiguration} von $M$ ist ein Element aus $Q \times \Sigma^*$. Falls $M$ in der Konfiguration $(q_i, w) \in Q \times \Sigma^*$ ist, so bedeutet es, dass $M$ aktuell im Zustand $q_i \in Q$ ist und noch den Suffix $w \in \Sigma^*$ des Eingabeworts zu lesen hat.

Die Konfiguration $(q_0, x) \in \{q_0\} \times \Sigma^*$ nennt man eine \textbf{Startkonfiguration} von $M$ auf $x$. Die Berechnung des Worts $x$ beginnt somit an der Startkonfiguration $q_0$.\\
\end{definition}

\begin{definition}
Eine \textbf{Endkonfiguration} von $M$ hat die Form $(q_i, \lambda) \in Q \times \{ \lambda \}$\\
\end{definition}

\begin{definition}
Ein \textbf{Schritt} von $M$ ist eine Relation $\vdash_M \subseteq (Q \times \Sigma^*) \times (Q \times \Sigma^*)$ und ist definiert durch
\[
(q, w) \vdash_{M} (p, x) \Leftrightarrow w = ax,\ a \in \Sigma \land \delta(q, a) = p
\]

Es handelt sich somit um eine Relation auf Konfigurationen des EA $M$. Es beschreibt den Übergang von einer Konfiguration in die nächste, nachdem der nächste Buchstabe von $w$ (hier der Buchstabe $a$) gelesen wurde.\\
\end{definition}

\begin{definition}
Eine \textbf{Berechnung} $C$ von $M$ ist eine endliche Folge $C = C_0, C_1, C_2, C_3, \ldots, C_n$ von Konfigurationen ($C_i \in (Q \times \Sigma^*),\ i=0 \ldots n$), so dass $C_i \vdash_M C_{i+1}$ gilt für alle $0 \leq i \leq n-1$.

Falls $C_0 = (q_0, x)$ und $C_n \in Q \times \{ \lambda \}$, so ist $C$  die Berechnung von $M$ auf einer Eingabe $x \in \Sigma^*$.

Falls $C_n \in F \times \{ \lambda \}$, so sagen wir, dass $C$ eine \textbf{akzeptierende Berechnung} von $M$ auf $x$ ist, und dass $M$ das Wort $x$ akzeptiert.

Falls $C_n \in (Q - F) \times \{ \lambda \}$, so ist $C$ eine \textbf{verwerfende Berechnung} von $M$ auf $x$, und dass $M$ das Wort $x$ verwirft.\\
\end{definition}

\begin{definition}
Die von $M$ akzeptierte Sprache $L(M)$ ist definiert als
\begin{align*}
L(M) := \{ w \in \Sigma^* | \text{ die Berechnung von $M$ auf $w$ endet in einer}\\ \text{Endkonfiguration $(q, \lambda)$ mit $q \in F$} \}
\end{align*}
\end{definition}

\begin{definition}
$\mathcal{L}(EA) = \{L(M) | M \text{ ist ein EA} \}$ ist die Klasse der Sprachen, die von endlichen Automaten akzeptiert werden. $\mathcal{L}(EA)$ bezeichnet man auch als die \textbf{Klasse der regulären Sprachen}, und jede Sprache $L$ aus $\mathcal{L}(EA)$ als \textbf{regulär}.\\
\end{definition}

\begin{definition}
Sei $M$ ein EA. Wir definieren $\vdash_M^*$ als die reflexive und transitive Hülle der Schrittrelation $\vdash_M$ von $M$. Somit gilt
\[
(q, w) \vdash_M^* (p, u) \Leftrightarrow (q = p \land w = u) \lor \exists k \in \N - \{0\} \text{ so dass}
\]
\begin{itemize}
  \item $w = a_1 a_2 a_3 a_4 \ldots a_k u, \ a_i \in \Sigma \ i = 1, \ldots, k$ und
  \item $\exists r_1, r_2, \ldots, r_{k-1} \in Q$, so dass
  $(q, w) \vdash_M (r_1, a_2\ldots a_k u) \vdash_M (r_2, a_3 \ldots a_k u) \vdash_M \ldots \vdash_M (r_{k-1}, a_k u) \vdash_M (p, u)$
\end{itemize}

$(q, w) \vdash_M^* (p, u)$ sagt also aus, dass es eine Berechnung von $M$ gibt, die ausgehen von der Konfiguration $(q, w)$ zur Konfiguration $(p, u)$ führt.\\
\end{definition}

\begin{definition}
Sei $\hat{\delta}: Q \times \Sigma^* \to Q$ definiert durch:
\begin{itemize}
  \item $\hat\delta(q, \lambda) = q$ für alle $q \in Q$
  \item $\hat\delta(q, wa) = \delta(\hat\delta(q, w), a)$ für alle $a \in \Sigma,\ w \in \Sigma^*,\ q \in Q$
\end{itemize}

Wenn $M$ im Zustand $q$ ist und das Wort $w$ zu lesen beginnt, dann bedeutet $\hat\delta(q, w) = p$ dass $M$ im Zustand $p$ enden wird. Oder anders gesagt, es gilt: $(q, w) \vdash_M^* (p, \lambda)$.

Gekürzt können wir nun $L(M)$ wie folgt definieren:
\[
L(M) = \{ w \in \Sigma^* \ |\ (q_0, w) \vdash_M^* (p, \lambda),\ p \in F \} = \{ w \in \Sigma^* \ |\ \hat\delta(q_0, w) \in F \}
\]\\
\end{definition}

\section{Simulation}
\begin{lemma}
Sei $\Sigma$ ein Alphabet und $M_1 = (Q_1, \Sigma, \delta_1, q_{01}, F_1), M_2 = (Q_2, \Sigma, \delta_2, q_{02}, F_2)$ zwei EA. Für $\circledcirc \in \{\cup, \cap, -\}$ existiert jeweils ein EA $M$ mit $L(M) = L(M_1) \circledcirc L(M_2)$.
\end{lemma}

\begin{proof}
Die Existenz von $L(M)$ basiert auf der Idee einer Konstruktion von $M$ in welchem $M_1$ und $M_2$ simuliert werden. Dabei sind die Zustände von $M$ Paare der Form $(q, p) \in Q_1 \times Q_2$.

Formale konstruktion von $M$: Sei $M = (Q, \Sigma, \delta, q_0, F_\circledcirc)$ und:
\begin{itemize}
  \item $Q = Q_1 \times Q_2$
  \item $q_0 = (q_{01}, q_{02})$
  \item $\forall q \in Q_1, p \in Q_2, a \in \Sigma:\, \delta((q, p), a) = (\delta_1(q, a), \delta_2(p, a))$
  \item F ist:
  \begin{itemize}
    \item Falls $\circledcirc = \cup$, dann ist $F = F_1 \times Q_2 \cup Q_1 \times F_2$
    \item Falls $\circledcirc = \cap$, dann ist $F = F_1 \times F_2$
    \item Falls $\circledcirc = -$, dann ist $F = F_1 \times (Q_2 - F_2)$
  \end{itemize}
\end{itemize}

Um nun zu beweisen, dass ein solcher EA $M$ für die Operation $\circledcirc$ existiert, reicht es zu zeigen dass folgende Gleichheit gilt: $\forall x \in \Sigma^*: \hat\delta((q_{01}, q_{02}), x) = (\hat\delta_1(q_{01}, x), \hat\delta_2(q_{02}, x))$. Der Beweis kann mittels Induktion über der Länge von $x$ geführt werden.
\end{proof}

Diese Entwurfsmethode kann dazu eingesetzt werden endliche Automaten für komplexere Sprachen zu bauen, in dem EA für einfachere Sprachen zusammengesetzt werden. So kann die Sprache $L = \{x \in \Sigma_\text{bool}^* |\ |x|_0 \mod 3 = 1, |x|_1 \mod 3 = 2\}$ aus den Sprachen $L_1 = \{x \in \Sigma_\text{bool}^* |\ |x|_0 \mod 3 = 1\}$ und $L_2 = \{x \in \Sigma_\text{bool}^* |\ |x|_1 \mod 3 = 2\}$ konstruiert werden.

\section{Beweise der Nichtexistenz}
\subsection{Lemma 3.3}
\begin{lemma}[Lemma 3.3]
Sei $A = (Q, \Sigma, \delta_A, q_0, F)$ ein EA. Seien weiter $x, y \in \Sigma^*, x \not= y$, so dass $(q_0, x) \vdash_A^* (p, \lambda) \land (q_0, y) \vdash_A^* (p, \lambda)$ für ein $p \in Q$. Dann gilt für jedes $z \in \Sigma^*:\, xz \in L(A) \Leftrightarrow yz \in L(A)$.
\end{lemma}

Anders ausgedrückt: Wenn wir zwei Wörter betrachten, die im EA $A$ im selben Zustand landen, so haben wir ab diesem Moment keine Information mehr, welches der beiden Wörter wir gelesen haben. Konkatinieren wir nun ein Wort $z \in \Sigma^*$ zu den beiden ausgewählten Wörtern, so müssen beide Wörter in $L(A)$ sein oder beide nicht.

Dieses Lemma kann dazu eingesetzt werden zu zeigen, dass eine Sprache nicht regulär ist ($L \not\in \mathcal{L}(EA)$). Dazu führt man den Beweis indirekt und nimmt an, dass $L$ regulär sei und wendet das Lemma an. Dabei wird festgestellt, dass es ein $z$ gibt, für welches $xz \in L$, aber $yz \not\in L$, was ein Wiederspruch ist.

\subsubsection{Beispiel}
Es sei zu zeigen, dass $L = \{0^n1^n | n \in \N\}$ nicht regulär ist.
\begin{proof}
Wir führen den Beweis indirekt mittels dem Lemma 3.3. Sei $L \in \mathcal{L}(EA)$. Somit existiert ein EA $A = (Q, \Sigma, \delta, q_0, F)$ mit $L(A) = L$.

Wir betrachten die Wörter $0^1, 0^2, \ldots, 0^{|Q|+1}$. Wir haben somit $|Q| + 1$ Wörter. Somit existieren $i, j \in \{1, 2, \ldots |Q| + 1\}, i < j$, so dass $\hat\delta(q_0, 0^i) = \hat\delta(q_0, 0^j)$. Nach Lemma 3.3 hat nun zu gelten: $\forall z \in \Sigma^*: 0^i z \in L \Leftrightarrow 0^j z \in L$. Dem ist aber nicht so. Für $z = 1^i$ erhalten wir: $0^i 1^i \in L$, aber auch $0^j 1^i \not\in L$ ($i < j$).

Somit ist $L \not\in \mathcal{L}(EA)$.
\end{proof}

\subsection{Pumping-Lemma}
\begin{lemma}[Pumping-Lemma für reguläre Sprachen]
Sei $L$ regulär. Dann existiert die Konstante $n_0 \in \N$, so dass jedes Wort $w \in \Sigma^*$ mit $|w| \geq n_0$ zerlegt werden kann in $w = y x z$, mit
\begin{itemize}
  \item $|yx| \leq n_0$,
  \item $|x| \geq 1$ und
  \item entweder $\{xy^kz | k \in \N\} \subseteq L$ oder $\{xy^kz | k \in \N \} \cap L = \emptyset$.
\end{itemize}

\end{lemma}

\subsubsection{Beispiel}
Es sei zu zeigen, dass $L = \{0^n 1^n | n \in \N\}$ nicht regulär ist.

\begin{proof}
Wir führen den Beweis indirekt mittels des Pumping-Lemmas. Sei $L \in \mathcal{L}(EA)$. Dann existiert eine Konstante $n_0$ mit den im Pumping-Lemma beschriebenen Eigenschaften. Wir betrachten das Wort $w = 0^{n_0} 1^{n_0}$. Es gilt offensichtlich, dass $|w| \geq n_0$. Somit kann $w$ zerlegt werden in $w = y x z$.

Bedingt durch die Eigenschaft $|yx| \leq n_0$ gilt hier, dass $y = 0^l$ und $x = 0^m$ für $l, m \in \N$. Bedingt durch die zweite Eigenschaft $|x| \geq 1$ ist $m \not= 0$. Nun prüfen wir ob mit diesen Eigenschaften, auch die dritte Pumping-Lemma Eigenschaft gilt. Wir erhalten also $\{ y x^k z | z \in \N\} = \{ 0^{n_0 - l} (0^m)^k 1^{n_0} | k \in \N \} = \{ 0^{n_0 - m + km} 1^{n_0} | k \in \N \}$. Wählen wir $k = 0$, so erhalten wir das Wort $0^{n_0 - m} 1^{n_0}$, was nicht in $L$ liegt, da $m \not= 0$. Für $k = 1$ hingegen liegt das Wort in $L$. Dies ist ein Wiederspruch. Somit ist $L$ nicht regulär.
\end{proof}

\subsection{Kolmogorov-Komplexität}
\todo[inline]{... to add ...}

\section{Nichtdeterminismus}
Determinismus bei einem deterministischen endlichen Automaten besagt, dass in jeder Konfiguration des EA eindeutig festgelegt ist, was im nächsten Schritt passiert. Somit bestimmt ein EA und ein Wort $x$ eindeutig die Berechnung von $x$ auf dem EA. Beim Nichtdeterminismus hingegen ist es erlaubt bei einer Konfiguration eine Auswahl an möglichen weiteren Schritten zu haben.

\begin{definition}
Ein \textbf{nichtdeterministischer endlicher Automat (NEA)} ist ein Quintupel $M = (Q, \Sigma, \delta, q_0, F)$ mit
\begin{itemize}
  \item $Q$ eine endliche Menge von Zuständen
  \item $\Sigma$ ist das Eingabealphabet
  \item $q_0 \in Q$ der Anfangszustand,
  \item $F \subseteq Q$ die Menge der akzeptierenden Zustände
  \item $\delta$ eine Funktion $\delta: Q \times \Sigma \to \mathcal{P}(Q)$. Die Übergangsfunktion.\\
\end{itemize}
\end{definition}

Man bemerkt, dass $Q, \Sigma, q_0, F$ die gleiche Bedeutungen haben bei einem EA wie auch bei einem NEA.\\

Der Unterschied liegt also in der $\delta$-Funktion, welche als Wertebereich nicht $Q$, sondern $\mathcal{P}(Q)$ hat. Darin liegt auch der zentrale Unterschied. Ein Übergang kann in beliebig vielen Zuständen enden, oder keinem.

Ein NEA akzeptiert ein Wort, falls es eine Berechnung gibt, die in einem Zustand $p \in F$ landet. Es ist dabei egal, wo alle alternativen Berechnungen enden. Es reicht, wenn eine in einem akzeptierenden Zustand landet.\\

\begin{definition}
Zur Übergangsfunktion $\delta$ wird $\hat\delta: Q \times \Sigma^* \to \mathcal{P}(Q)$ wie folgt definiert:
\begin{itemize}
  \item $\hat\delta(q, \lambda) = \{q\} \forall q \in Q$
  \item $\hat\delta(q, wa) = \{p | \text{ es existiert ein } r \in \hat\delta(q, w), \text{ so dass } p \in \delta(r, a)\} = \bigcup_{r \in \hat\delta(q, w)} \delta(r, a)$\\
\end{itemize}

\end{definition}

Aus der Definition sieht man, dass $\delta(q, w)$ jeweils die Menge aller Zustände aus $Q$ liefert, die aus $q \in Q$ durch das vollständige Lesen von $w$ erreichbar sind. Somit erhalten wir für einen NEA $M$: $L(M) = \{ w \in \Sigma^* |\ \hat\delta(q_0, w) \cap F \not= \emptyset \}$.\\

\begin{satz}
Zu jedem NEA \( M \) existiert ein EA \( A \), so dass \(L(M) = L(A)\).
\end{satz}

\begin{proof}
Sei \( M = (Q, \Sigma, \delta_M, q_0, F) \) ein NEA und \( A = (Q_A, \Sigma_A, \delta_A, q_{0A}, F_A) \) ein EA. \(A\) können wir dabei aus \(M\) wie folgt konstruieren:
\begin{itemize}
  \item \(Q_A = \{ \langle P \rangle \ |\ P \subset Q \}\)
  \item \(\Sigma_A = \Sigma\)
  \item \(q_{0A} = \langle \{q_0\} \rangle\)
  \item \(F_A = \{ \langle P \rangle \ |\ P \subseteq Q, P \cap F \not= \emptyset \}\)
  \item \( \delta_A: Q_A \times \Sigma_A \to Q_A \) und definiert wie folgt: \[ \forall \langle P \rangle \in Q_A, \forall a \in \Sigma_A: \delta_A(\langle P \rangle, a) = \left\langle \bigcup_{p \in P} \delta_M (p, a) \right\rangle
  \]\\
\end{itemize}

\end{proof}